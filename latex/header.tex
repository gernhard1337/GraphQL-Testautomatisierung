% Diese Datei enthält Einstellungen und Paketimporte.
% Zur besseren Ordnung sollten Sie weitere Einstellungen bzw. Importe dieser Datei anfügen.

% Paketimporte
\documentclass[11pt,a4paper,german]{scrreprt}
\usepackage[utf8]{inputenc}  % Encoding
\usepackage{babel}  % Unterstützung verschiedener Sprachen
\usepackage{graphicx}  % Einbindung von Grafiken
\usepackage{amsmath}  % Verschiedene Symbole
\usepackage{amssymb}  % Verschiedene Symbole
\usepackage{latexsym}  % Verschiedene Symbole
\usepackage{booktabs}  % Schönere Tabellen
\usepackage{multicol}  % Flexiblere Tabellen
\usepackage[printonlyused]{acronym}  % Abkürzungsverwaltung
\usepackage{hyperref}  % Klickbare URLs, Titel und PDF-Metadaten
\usepackage{lipsum}  % Platzhaltertext (kann entfernt werden)
\usepackage[backend=bibtex,style=ieee]{biblatex}  % Moderne Literaturverwaltung
\usepackage{tikz} % Graphen zeichnen
\usetikzlibrary{positioning,shapes,shadows,arrows}

% Einstellungen
\hypersetup{
    colorlinks,
    citecolor=black,
    filecolor=black,
    linkcolor=black,
    urlcolor=black
}

\makeatletter

\newcommand\frontmatter{
	\cleardoublepage
	\pagenumbering{roman}}

\newcommand\mainmatter{
	\cleardoublepage
	\pagenumbering{arabic}}

\newcommand\backmatter{
  \if@openright
    \cleardoublepage
  \else
    \clearpage
  \fi
   }

\makeatother

% Laden der Literaturdaten
\addbibresource{quellen.bib}

% Definition der Dokumentendaten
% Zur besseren Kapselung wurde die Eingabe Ihrer Daten in diese Datei ausgelagert.
% Bitte einfach ausfüllen.

\newcommand{\TitelDeutsch}{Integrationstesten von GraphQL mittels Prime-Path Abdeckung} % Hier zentral angeben
\newcommand{\TitelEnglisch}{Integration testing of GraphQL using Prime-Path Coverage} % Übersetzten Titel auch angeben
\newcommand{\ArbeitsTyp}{Masterarbeit} % unzutreffendes hier löschen
\newcommand{\AuthorName}{Tom Lorenz} % Hier ändern
\newcommand{\MatrikelNr}{3711679}
\newcommand{\Studiengang}{Informatik M. Sc.}
\newcommand{\DatumThemenausgabe}{16.05.2023}
\newcommand{\DatumAbgabe}{31.8.2023}
\newcommand{\BetreuerEins}{Prof. Dr. rer. nat. Leen Lambers}
\newcommand{\BetreuerZwei}{Prof. Dr. rer. nat. Gerd Wagner} % Bitte immer akademische Titel angeben
\newcommand{\GutachterEins}{M. Sc. Lucas Sakizloglou} % Bitte immer akademische Titel angeben
  % In dieser Datei die Daten der Arbeit eingeben.

\titlehead{{Brandenburgische Technische Universität Cottbus-Senftenberg\\ Institut für Informatik\\ Fachgebiet Praktische Informatik/Softwaresystemtechnik}}
\subject{Masterarbeit\\ \vspace{0.5cm}\includegraphics[width=0.5\textwidth]{img/btu-logo.pdf}}
\title{\TitelDeutsch \\ {\normalsize \TitelEnglisch}}
\author{\AuthorName \\ MatrikelNr.: \MatrikelNr \\Studiengang: \Studiengang}
\date{\textit{Datum der Themenausgabe: \DatumThemenausgabe} \\ \textit{Datum der Abgabe: \DatumAbgabe}}
\publishers{1. Betreuer: \BetreuerEins \\ 2. Betreuer: \BetreuerZwei}

\hypersetup{pdfauthor={\AuthorName}}
\hypersetup{pdftitle={\TitelDeutsch}}

\tikzstyle{abstract}=[rectangle, draw=black, rounded corners, fill=blue!40, drop shadow,
text centered, anchor=north, text=white, text width=3cm]
\tikzstyle{comment}=[rectangle, draw=black, rounded corners, fill=green, drop shadow,
text centered, anchor=north, text=white, text width=3cm]
\tikzstyle{myarrow}=[->, >=open triangle 90, thick]
\tikzstyle{line}=[-, thick]