% Diese Datei enthält Einstellungen und Paketimporte.
% Zur besseren Ordnung sollten Sie weitere Einstellungen bzw. Importe dieser Datei anfügen.

% Paketimporte
\documentclass[11pt,a4paper,german]{scrreprt}
\usepackage[utf8]{inputenc}  % Encoding
\usepackage{babel}  % Unterstützung verschiedener Sprachen
\usepackage{graphicx}  % Einbindung von Grafiken
\usepackage{amsmath}  % Verschiedene Symbole
\usepackage{tabularx}
\usepackage{ragged2e}
\usepackage{amssymb}  % Verschiedene Symbole
\usepackage{latexsym}  % Verschiedene Symbole
\usepackage{booktabs}  % Schönere Tabellen
\usepackage{multicol}  % Flexiblere Tabellen
\usepackage[printonlyused]{acronym}  % Abkürzungsverwaltung
\usepackage{hyperref}  % Klickbare URLs, Titel und PDF-Metadaten
\usepackage{lipsum}  % Platzhaltertext (kann entfernt werden)
\usepackage[backend=bibtex,style=ieee]{biblatex}  % Moderne Literaturverwaltung
\usepackage{tikz} % Graphen zeichnen

\usetikzlibrary{positioning,shapes,shadows,arrows}

% Einstellungen
\hypersetup{
    colorlinks,
    citecolor=black,
    filecolor=black,
    linkcolor=black,
    urlcolor=black
}

\makeatletter

\newcommand\frontmatter{
	\cleardoublepage
	\pagenumbering{roman}}

\newcommand\mainmatter{
	\cleardoublepage
	\pagenumbering{arabic}}

\newcommand\backmatter{
  \if@openright
    \cleardoublepage
  \else
    \clearpage
  \fi
   }

\usepackage{listings}
\usepackage{xcolor}

\colorlet{punct}{red!60!black}
\definecolor{background}{HTML}{EEEEEE}
\definecolor{delim}{RGB}{20,105,176}
\colorlet{numb}{magenta!60!black}

\lstdefinelanguage{json}{
    basicstyle=\normalfont\ttfamily,
    numbers=left,
    numberstyle=\scriptsize,
    stepnumber=1,
    numbersep=8pt,
    showstringspaces=false,
    breaklines=true,
    frame=lines,
    backgroundcolor=\color{background},
    literate=
    *{0}{{{\color{numb}0}}}{1}
        {1}{{{\color{numb}1}}}{1}
        {2}{{{\color{numb}2}}}{1}
        {3}{{{\color{numb}3}}}{1}
        {4}{{{\color{numb}4}}}{1}
        {5}{{{\color{numb}5}}}{1}
        {6}{{{\color{numb}6}}}{1}
        {7}{{{\color{numb}7}}}{1}
        {8}{{{\color{numb}8}}}{1}
        {9}{{{\color{numb}9}}}{1}
        {:}{{{\color{punct}{:}}}}{1}
        {,}{{{\color{punct}{,}}}}{1}
        {\{}{{{\color{delim}{\{}}}}{1}
        {\}}{{{\color{delim}{\}}}}}{1}
        {[}{{{\color{delim}{[}}}}{1}
        {]}{{{\color{delim}{]}}}}{1},
}

% GraphQL-Syntax-Hervorhebung
\lstdefinelanguage{GraphQL}{
    keywords={query, mutation, subscription, fragment, on, true, false, null},
    sensitive=true,
    comment=[l]{\#},
    morecomment=[s]{/*}{*/},
    string=[b]",
    morestring=[b]',
    literate=
        {<=}{{$\leq$}}2
        {>=}{{$\geq$}}2
        {!=}{{$\neq$}}2
        {==}{{$=$}}2
        {&&}{{\&\&}}2
        {||}{{$\mid\mid$}}2
        {...}{{$\ldots$}}2
%{_}{{$\lambda$}}1
        {\\\\}{{\char`\\\char`\\}}1
        {\\n}{{\char`\\n}}1
        {\\t}{{\char`\\t}}1
        {\\u2028}{{\char`\\u2028}}1
        {\\u2029}{{\char`\\u2029}}1
        {\\v}{{\char`\\v}}1
        {\\b}{{\char`\\b}}1
        {\\r}{{\char`\\r}}1
        {\\f}{{\char`\\f}}1,
}

% Einstellungen für den Code-Stil
\lstset{
    basicstyle=\ttfamily,
    keywordstyle=\color{blue},
    commentstyle=\color{gray},
    stringstyle=\color{orange},
    numbers=left,
    numberstyle=\scriptsize\color{gray},
    frame=single,
    breaklines=true,
    showstringspaces=false,
    captionpos=b,
}


\lstdefinelanguage{Python}{
    keywords={False, None, True, and, as, assert, break, class, continue, def, del, elif, else, except, finally, for, from, global, if, import, in, is, lambda, nonlocal, not, or, pass, raise, return, try, while, with, yield},
    keywordstyle=\color{blue}\bfseries,
    identifierstyle=\color{black},
    commentstyle=\color{gray}\itshape,
    stringstyle=\color{orange},
    morestring=[b]',
    morestring=[b]",
    morecomment=[s]{"""}{"""},
    morecomment=[l]{\#},
    morecomment=[s]{'''}{'''},
    sensitive=true,
    showstringspaces=false,
}


\definecolor{backcolour}{rgb}{0.95,0.95,0.92}
\definecolor{codegreen}{rgb}{0,0.6,0}
\definecolor{codegray}{rgb}{0.5,0.5,0.5}
\definecolor{codepurple}{rgb}{0.58,0,0.82}
\definecolor{codered}{rgb}{0.64,0.08,0.08}

\lstdefinelanguage{javascript}{
    keywords={const, let, break, case, catch, continue, debugger, default, delete, do, else, finally, for, function, if, in, instanceof, new, return, switch, this, throw, try, typeof, var, void, while, with, yield, import, export, async, await},
    keywordstyle=\color{blue}\bfseries,
    ndkeywords={class, enum, interface, extends, super, public, static, getter, setter},
    ndkeywordstyle=\color{codered}\bfseries,
    identifierstyle=\color{black},
    sensitive=false,
    comment=[l]{//},
    morecomment=[s]{/*}{*/},
    commentstyle=\color{codegreen}\ttfamily,
    stringstyle=\color{codepurple}\ttfamily,
    morestring=[b]',
    morestring=[b]"
}




% Laden der Literaturdaten
\addbibresource{quellen.bib}
\newtheorem{definition}{Definition}
% Definition der Dokumentendaten
% Zur besseren Kapselung wurde die Eingabe Ihrer Daten in diese Datei ausgelagert.
% Bitte einfach ausfüllen.

\newcommand{\TitelDeutsch}{Integrationstesten von GraphQL mittels Prime-Path Abdeckung} % Hier zentral angeben
\newcommand{\TitelEnglisch}{Integration testing of GraphQL using Prime-Path Coverage} % Übersetzten Titel auch angeben
\newcommand{\ArbeitsTyp}{Masterarbeit} % unzutreffendes hier löschen
\newcommand{\AuthorName}{Tom Lorenz} % Hier ändern
\newcommand{\MatrikelNr}{3711679}
\newcommand{\Studiengang}{Informatik M. Sc.}
\newcommand{\DatumThemenausgabe}{16.05.2023}
\newcommand{\DatumAbgabe}{31.8.2023}
\newcommand{\BetreuerEins}{Prof. Dr. rer. nat. Leen Lambers}
\newcommand{\BetreuerZwei}{Prof. Dr. rer. nat. Gerd Wagner} % Bitte immer akademische Titel angeben
\newcommand{\GutachterEins}{M. Sc. Lucas Sakizloglou} % Bitte immer akademische Titel angeben
  % In dieser Datei die Daten der Arbeit eingeben.

\titlehead{{Brandenburgische Technische Universität Cottbus-Senftenberg\\ Institut für Informatik\\ Fachgebiet Praktische Informatik/Softwaresystemtechnik}}
\subject{Masterarbeit\\ \vspace{0.5cm}\includegraphics[width=0.5\textwidth]{img/btu-logo}}
\title{\TitelDeutsch \\ {\normalsize \TitelEnglisch}}
\author{\AuthorName \\ MatrikelNr.: \MatrikelNr \\Studiengang: \Studiengang}
\date{\textit{Datum der Themenausgabe: \DatumThemenausgabe} \\ \textit{Datum der Abgabe: \DatumAbgabe}}
\publishers{ Betreuer 1: \BetreuerEins \\ Betreuer 2: \BetreuerZwei \\ Gutachter: \GutachterEins}

\hypersetup{pdfauthor={\AuthorName}}
\hypersetup{pdftitle={\TitelDeutsch}}

\tikzstyle{abstract}=[rectangle, draw=black, rounded corners, fill=blue!40, drop shadow,
text centered, anchor=north, text=white, text width=3cm]
\tikzstyle{comment}=[rectangle, draw=black, rounded corners, fill=green, drop shadow,
text centered, anchor=north, text=white, text width=3cm]
\tikzstyle{myarrow}=[->, >=open triangle 90, thick]
\tikzstyle{line}=[-, thick]