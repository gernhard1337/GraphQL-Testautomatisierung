\section{Testen}

Indem technische Geräte und somit auch Software im umfangreichen Maßstab Einzug nehmen in nahezu alle Bereiche des
Lebens ist es wichtig die Sicherheit, Qualität und Zuverlässigkeit von Software sicherzustellen.
Um all dies sicherzustellen sind strukturelle Tests von Software nötig.
Ziel ist es sicherzustellen, dass die Software den definierten Anforderungen und Spezifikation entspricht.
Hierbei sind diverse Techniken und Ansätze verfolgt.
Im Rahmen dieser Arbeit wird sich insbesondere auf Integrationstests fokussiert also Tests, die Zusammenarbeit von einzelnen Softwarekomponenten (Units)
sicherstellen sollen.

\subsection{Arten von Tests}

Es existieren verschiedene Arten von Tests welche unterschiedliche Ziele haben.
Nachfolgend wird eine Auswahl verschiedener Arten erklärt wobei wir in der Granularität aufsteigen, wir fangen mit feingranularen Unit-Tests an
und steigen immer weiter zu kompletten System-Tests auf.
In der Mitte davon liegt das Integrations-Testen.
Diese Methode wird mit dieser Arbeit für einen Anwendungsfall benötigt.
Die Erklärung der anderen Arten erfolgt für eine einfachere Einordnung ebendieser.

\subsubsection{Unit Testing}

Feingranulare Tests die zur Aufgabe haben, einzelne Funktionen/Methoden zu testen.
Es wird insbesondere darauf geachtet, dass die einzelne Methode genau das macht, was Sie soll.
Es existieren diverse Tools, die diesen Schritt vereinfachen.
Diese Tools helfen einem dabei, dass man Tests definieren kann \& ein erwartetes Verhalten der Funktion angibt.
Eine Auswertung dieser Funktionalität wird dann vom Tool übernommen und bei etwaiigen Fehlern gibt es eine Erklärung
nach dem Muster: \vern+ABC+ war erwartet, \verb+DEF+ ist eingetreten.
Unit-Tests werden bei der Entwicklung der einzelnen Funktionen von der Software entwickelt (sollten sie zumindest).
Im Idealfall wird Test-Driven-Development verwendet, hierbei wird erst der Test entwickelt und dann die Methode entwickelt,
die diese Tests erfüllen muss.

\subsubsection{Integration Testing}

Eine Granularitätsebene höher sind die Integrationstests.
Hier wird getestet, ob einzelne Komponenten der Software miteinander gut zusammen arbeiten, d.h.\ ob sie integer sind.
Die Testentwicklung ist hierbei meist eher komplex da einzelne Komponenten der Software selbst eine sehr hohe Komplexität haben können.
Eine automatische Testentwicklung ist auf dieser Granularitätseben eigentlich eher selten der Fall, allerdings bietet sich
im Kontext von GraphQL die Automatisierung durchaus an da die Zusammenarbeit von Softwarekomponenten mit klar definierter Struktur
lohnenswert scheint.
Will man allerdings die Intergration von großen Librarys als einzelne Komponente in der eigenen Anwendung testen, so gestaltet sich dies
meist schwer und muss manuell getestet werden da meist komplexe Datenstrukturen händisch gemockt (Mocken als Kapitel oder Glossar?) werden müssen.

\subsubsection{System Testing}

Auf der Ebene des System-Testing wird das komplette entwickelte System getestet.
Hierbei ist sicherzustellen, dass alle funktionalen Anforderungen an das System eingehalten werden.
Die Tests werden hierbei so realistisch wie möglich ausgeführt, d.h. das Testsystem soll möglichst nah am späteren Produktivsystem sein
und Testfälle sollen die funktionalen Anforderungen der Software abdecken.
So sind insbesondere alle Geschäftsprozesse zu testen.
Im allgemeinen werden System-Tests als Blackbox Tests durchgeführt, dies bedeutet, dass nur externe Funktionsmerkmale getest werden,
z.B. ob eine gewünschte Interaktion stattfand.
Hierbei wird keine Rücksicht auf interne Zustände genommen.
Ziel ist es, die Funktionsweise so zu testen wie ein realer Benutzer die Software nutzen würde.

\subsection{Test-Coverage}

Es ist nun bekannt, welche Arten des Testens es gibt.
Allerdings ist noch nicht klar, wie viele Tests ausreichend sind.
Man könnte nun argumentieren, dass man einfach jede einzelne Eingabe in einem Programm testen könnte um zu sehen,
dass für jede Eingabe die korrekte Ausgabe kommt.
Hierbei bemerkt man jedoch relativ schnell, dass dies mit heutigen Prozessoren nicht mehr möglich ist.
Als Beispiel sei hier eine simple Addition von 2 64-bit Integern genannt.
Für eine komplette Testung dieser simplen Addition gibt es \verb+2^64 ≈ 18 Trillionen+ Kombinationen.
Mit einem 3GHz Prozessor wäre eine vollständige Testung nach \verb+2^64 / 3.000.000.000 ≈ 6.149.571+ Sekunden (69 Tage) erledigt.
Es ist also ersichtlich, dass schon so eine vermeintlich einfache Funktion nicht komplett testbar ist.
Daher wird also ein strukturierter Ansatz benötigt, der den Testraum einerseits klein hält, andererseits trotzdem dafür sorgt, dass
Fehler ausgeschlossen werden können.
Hierfür wurden formale Coverage-Kriterien entwickelt.(vgl. S. 17 software-testing)




\subsubsection{Coveragekriterien}










