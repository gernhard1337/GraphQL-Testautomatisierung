\section{Testen}

Indem technische Geräte und somit auch Software im umfangreichen Maßstab Einzug nehmen in nahezu alle Bereiche des
Lebens ist es wichtig die Sicherheit, Qualität und Zuverlässigkeit von Software sicherzustellen.
Um all dies sicherzustellen sind strukturelle Tests von Software nötig.
Ziel ist es sicherzustellen, dass die Software den definierten Anforderungen und Spezifikation entspricht.
Hierbei sind diverse Techniken und Ansätze verfolgt.
Im Rahmen dieser Arbeit wird sich insbesondere auf Integrationstests fokussiert also Tests, die Zusammenarbeit von einzelnen Softwarekomponenten (Units)
sicherstellen sollen.

\subsection{Typen von Tests}

Es gibt verschiedene Sichtweisen auf das zu testende System.
Die Sichtweisen können intern oder extern bestimmt sein.
So sind Faktoren wie möglicher Zugriff auf Quellcode oder Architekturdetails des Systems essentiell wichtig um
zu entscheiden, welche Art von Tests angewandt werden.
Es gibt zwei generelle Typen von Sichtweisen und eine Mischform.
Wir betrachten hierbei das zu testende System als Box.
Diese Box kann entweder transparent gesehen werden, halbtransparent oder untransparent.
Der Transparente Ansatz nennt sich WhiteBox-Testing, der halbtransparente GreyBox-Testing und der Untransparente BlackBox-Testing.
Die Ansätze unterscheiden sich vorallem in den zur Verfügung stehenden Informationen des Systems.
Wir vernachlässigen hierbei das GreyBox-Testing da dieses keine Relevanz für das hier zugrunde liegende Problem besitzt.

\subsubsection{White-Box Testing}

Im WhiteBox-Testing stehen alle Informationen über das System zur Verfügung.
Der Tester hat Zugriff auf Code, Architekturdetails und besitzt Kenntniss über alle möglichen Details des Systems.
Somit kann der Tester auf alle möglichen Informationen über das System zugreifen und damit seine Tests generieren.
Die erstellten Tests fundieren dann auf einem soliden Niveau, dass begründet wird durch das Domänenwissen über das System.

\subsubsection{Black-Box Testing}

Der Idealfall des Testens ist zwar der WhiteBox-Test allerdings ist es durch verschiedene Limitierungen nicht immer möglich oder auch gewollt,
auf alle Informationen zuzugreifen.
Im BlackBox-Testing hat der Tester keinen Zugriff auf interne Funktionsweisen der Software.
Schwerpunkt des Testens ist es, dass die Software das tut was sie soll.
Wir interessieren uns hier nicht dafür was intern im System abläuft sondern nur, ob Eingabe zu Ausgabe passt.
In dieser Arbeit werden wir uns vor allem auf das BlackBox-Testing beziehen.
Dies folgt einerseits aus dem methodischen bezug zu \textit{Property-based Testing}\cite{property-based-testing}
als auch aus dem Fakt, dass das Testtool unabhängig von Implementierungsdetails nutzbar sein soll.

\subsection{Arten von Tests}

Es existieren verschiedene Arten von Tests welche unterschiedliche Ziele haben.
Nachfolgend wird eine Auswahl verschiedener Arten erklärt wobei wir in der Granularität aufsteigen, wir fangen mit feingranularen Unit-Tests an
und steigen immer weiter zu kompletten System-Tests auf.
In der Mitte davon liegt das Integrations-Testen.
Diese Methode wird mit dieser Arbeit für einen Anwendungsfall benötigt.
Die Erklärung der anderen Arten erfolgt für eine einfachere Einordnung ebendieser.

\subsubsection{Unit Testing}

Feingranulare Tests die zur Aufgabe haben, einzelne Funktionen/Methoden zu testen.
Es wird insbesondere darauf geachtet, dass die einzelne Methode genau das macht, was Sie soll.
Es existieren diverse Tools, die diesen Schritt vereinfachen.
Diese Tools helfen einem dabei, dass man Tests definieren kann \& ein erwartetes Verhalten der Funktion angibt.
Eine Auswertung dieser Funktionalität wird dann vom Tool übernommen und bei etwaiigen Fehlern gibt es eine Erklärung
nach dem Muster: \vern+ABC+ war erwartet, \verb+DEF+ ist eingetreten.
Unit-Tests werden bei der Entwicklung der einzelnen Funktionen von der Software entwickelt (sollten sie zumindest).
Im Idealfall wird Test-Driven-Development verwendet, hierbei wird erst der Test entwickelt und dann die Methode entwickelt,
die diese Tests erfüllen muss.

\subsubsection{Integration Testing}

Eine Granularitätsebene höher sind die Integrationstests.
Hier wird getestet, ob einzelne Komponenten der Software miteinander gut zusammen arbeiten, d.h.\ ob sie integer sind.
Die Testentwicklung ist hierbei meist eher komplex da einzelne Komponenten der Software selbst eine sehr hohe Komplexität haben können.
Eine automatische Testentwicklung ist auf dieser Granularitätseben eigentlich eher selten der Fall, allerdings bietet sich
im Kontext von GraphQL die Automatisierung durchaus an da die Zusammenarbeit von Softwarekomponenten mit klar definierter Struktur
lohnenswert scheint.
Will man allerdings die Intergration von großen Librarys als einzelne Komponente in der eigenen Anwendung testen, so gestaltet sich dies
meist schwer und muss manuell getestet werden da meist komplexe Datenstrukturen händisch gemockt (Mocken als Kapitel oder Glossar?) werden müssen.

\subsubsection{System Testing}

Auf der Ebene des System-Testing wird das komplette entwickelte System getestet.
Hierbei ist sicherzustellen, dass alle funktionalen Anforderungen an das System eingehalten werden.
Die Tests werden hierbei so realistisch wie möglich ausgeführt, d.h. das Testsystem soll möglichst nah am späteren Produktivsystem sein
und Testfälle sollen die funktionalen Anforderungen der Software abdecken.
So sind insbesondere alle Geschäftsprozesse zu testen.
Im allgemeinen werden System-Tests als Blackbox Tests durchgeführt, dies bedeutet, dass nur externe Funktionsmerkmale getest werden,
z.B. ob eine gewünschte Interaktion stattfand.
Hierbei wird keine Rücksicht auf interne Zustände genommen.
Ziel ist es, die Funktionsweise so zu testen wie ein realer Benutzer die Software nutzen würde.

\subsection{Test-Coverage}

Es ist nun bekannt, welche Arten des Testens es gibt.
Allerdings ist noch nicht klar, wie viele Tests ausreichend sind.
Man könnte nun argumentieren, dass man einfach jede einzelne Eingabe in einem Programm testen könnte um zu sehen,
dass für jede Eingabe die korrekte Ausgabe kommt.
Hierbei bemerkt man jedoch relativ schnell, dass dies mit heutigen Prozessoren nicht mehr möglich ist.
Als Beispiel sei hier eine simple Addition von 2 64-bit Integern genannt.
Für eine komplette Testung dieser simplen Addition gibt es \verb+2^64 ≈ 18 Trillionen+ Kombinationen.
Mit einem 3GHz Prozessor wäre eine vollständige Testung nach \verb+2^64 / 3.000.000.000 ≈ 6.149.571+ Sekunden (69 Tage) erledigt.
Es ist also ersichtlich, dass schon so eine vermeintlich einfache Funktion nicht komplett testbar ist.
Daher wird also ein strukturierter Ansatz benötigt, der den Testraum einerseits klein hält, andererseits trotzdem dafür sorgt, dass
Fehler ausgeschlossen werden können.
Hierfür wurden formale Coverage-Kriterien entwickelt.\cite[vgl. S.17]{software-testing}

\subsubsection{Coveragekriterien}

Wie gezeigt ist ein "vollständiges Testen" - also ein ausprobieren aller Möglichkeiten einfach unmöglich.
Hierdurch sind wir gezwungen einen anderen Ansatz zu verfolgen.
Coveragekriteren liefern hierbei einen Ansatz die einem dabei helfen können, sinnvolle Tests zu entwickeln und zu entscheiden,
wann genug Tests entwickelt wurden.
Die Grundidee ist hierbei, dass wir Test-Requirements definieren.
Ein Test Requirement ist ein spezielles Element eines Software-Artefakts, dass von einem Testfall erfüllt sein muss.~\cite[vgl. S.17]{software-testing}
Hiermit ist gemeint, dass ein Test-Requirement ein spezifisches Kriterium definiert, dass durch einen Test überprüfbar
wird.
Ein Beispiel für ein Test-Requirement wäre es, das wir prüfen, ob die Division von 2 und 1 wirklich 2 ergibt.
Ein Coveragekriterium ist dann eine Regel beziehungsweise eine Sammlung von Regeln die eine Menge an Test-Requirements erzeugen.~\cite[vgl. S.17]{software-testing}
Bleiben wir bei dem eben genannten Beispiel wäre ein Coveragekriterium hierbei, dass wir die Divison durch 0 auch noch überprüfen
da diese speziell ausgeschlossen wird.
Es gibt diverse Methoden wie ein Coveragekriterium entwickelt werden kann.
Einige grundlegende Typen wären zum Beispiel Call-Coverage, Branch-Coverage, Boundary-Coverage oder Statement-Coverage.
(Man könnte hier noch weiter darauf eingehen aber das wäre nur um mehr Text zu haben TODO)







