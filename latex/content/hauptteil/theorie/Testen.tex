\section{Testen}

Indem technische Geräte und somit auch Software im umfangreichen Maßstab Einzug nehmen in nahezu alle Bereiche des
Lebens ist es wichtig die Sicherheit, Qualität und Zuverlässigkeit von Software sicherzustellen.
Um all dies sicherzustellen sind strukturelle Tests von Software nötig.
Ziel ist es sicherzustellen, dass die Software den definierten Anforderungen und Spezifikation entspricht.
Hierbei sind diverse Techniken und Ansätze verfolgt.
Im Rahmen dieser Arbeit wird sich insbesondere auf Integrationstests fokussiert also Tests, die Zusammenarbeit von einzelnen Softwarekomponenten (Units)
sicherstellen sollen.

\subsection{Arten von Tests}

\subsubsection{Acceptance Testing}

\subsubsection{System Testing}

\subsubsection{Integration Testing}

\subsubsection{Module Testing}

\subsubsection{Unit Testing}

\subsection{Test-Coverage}

\subsection{Test-Coverage & Graphen}

\subsection{Graphcoverage & Graphcoverage Kriterien}

\subsubsection{Edge Coverage}

\subsubsection{Node Coverage}

\subsubsection{Edge-Pair Coverage}

\subsubsection{Prime-Path Coverage}

\subsection{Graphcoverage für Code}

\subsection{Graphcoverage für GraphQL}

\subsection{Prime-Path Coverage Algorithmus}


