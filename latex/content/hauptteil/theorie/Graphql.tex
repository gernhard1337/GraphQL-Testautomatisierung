\section{GraphQL}
\label{graphql}

GraphQL~\cite{graphqlspecification} ist eine Open-Source Query-Language (Abfragesprache) und Laufzeitumgebung, die von Meta (ehemals Facebook) entwickelt wurde~\cite[vgl. Introduction]{graphqlspecification}.
Die Besonderheiten von GraphQL sind, dass mit nur einer einzelnen Anfrage mehrere Ressourcen gleichzeitig abgefragt werden können~\cite[vgl. No More Over- and Underfetching]{howtographql} und die
Daten in einem Schema durch einen Typgraphen definiert sind~\cite[vgl. Benefits of a Schema \& Type System]{howtographql}.
So lässt sich die Effizienz stark erhöhen, indem weniger Anfragen gestellt werden, welche zeitgleich eine höhere Informationsdichte haben.
GraphQL erleichtert außerdem die Kommunikation von Schnittstellen, indem die gewünschten Felder schon in der Query definiert sind und direkt den erwarteten Datentyp zusichern.
Hier liegt auch der große Vorteil im Vergleich zum technologischen Konkurrenten  REST API~\cite[vgl. Welche REST-Einschränkungen versucht GraphQL zu überwinden?]{awsrestgraphql}.
Bei REST-APIs sind für verschiedene Ressourcen jeweils eigene Anfragen nötig~\cite[vgl. No More Over- and Underfetching]{howtographql} und die Typsicherheit ist nicht so stark gegeben
wie bei GraphQL-APIs~\cite[vgl. Zusammenfassung der Unterschiede: REST vs. GraphQL]{awsrestgraphql}.
Diese beiden großen Vorteile sorgen dafür, dass GraphQL an Popularität gewinnt und zunehmend eingesetzt wird~\cite[vgl. Continued growth and the road ahead]{graphql-growing-report}.
\\
\\
GraphQL ist eine Abfragesprache und Spezifikation, dies bedeutet, dass GraphQL selbst keine konkrete Implementierung für eine Schnittstelle ist.
Implementierungen der GraphQL-Spezifikation sind in GraphQL-Servern umgesetzt, die in verschiedenen Programmiersprachen existieren.
Eine umfassende Auswahl verschiedenster Implementierungen findet sich in~\cite{gqlimplementation}.
Besonderer Beliebtheit erfreuen sich ApolloServer~\cite{apolloqgl}, Express GraphQL~\cite{expressgql} und HyGraph \cite{hygql}.
Da jedoch alle Server die GraphQL-Spezifikation umsetzen müssen und die hier entwickelten Tests aus GraphQL Anfragen bestehen, ist es irrelevant, welche konkrete GraphQL-Serverimplementierung das zu testende System verwendet.
Im Kontext dieser Arbeit ist ein tiefgreifendes, technologisches Verständnis von GraphQL essenziell.

\subsection{Schema \& Typen}
\label{schematypes}

Grundlage jeder GraphQL-API ist ein GraphQL-Schema~\cite[vgl. Core Concepts]{graphqlspecification}.
Das Schema definiert exakt, wie die Daten in der API aufgebaut sind und welche Informationen existieren~\cite[vgl. 3.2 Schema]{graphqlspecification}.
Ein GraphQL-Schema ist eine Sammlung von einzigartigen Typen und definiert die Einstiegspunkte der API.
Es gibt drei Einstiegspunkte: $query$ zum Daten abfragen, $mutation$ zum Daten verändern und $subscription$ um über Datenänderungen informiert zu werden~\cite[vgl. 3.2.1 Root Operation Types]{graphqlspecification}.
$query$ ist dabei als einziger verpflichtend, die anderen sind optional~\cite[vgl. 3.2.1]{graphqlspecification}.
Die drei Einstiegspunkte sind als Typen definiert.
Ein Typ ist die fundamentale Einheit eines jeden GraphQL-Schemas~\cite[vgl. 3.4 Types]{graphqlspecification}.
\newpage
Es gibt 6 verschiedene Typdefinitionen, diese sind:

\newcolumntype{C}{>{\Centering\arraybackslash}X}
\begin{center}
    \begin{table}[!ht]
        \begin{tabularx}{\textwidth}{|C|C|}
            \hline
            \textbf{ Typ } & \textbf{ Beschreibung} \\
            \hline
            \textbf{ ScalarTypeDefinition } & Primitive Datentypen, welche keine Verbindungen zu anderen Typen haben dürfen (Strings, Integer, \ldots). \\
            \hline
            \textbf{ ObjectTypeDefinition  } & Komplexere Datentypen, die Verbindungen untereinander haben und in Feldern ihre Verbindungen definieren. \\
            \hline
            \textbf{ InterfaceTypeDefinition } & Ein abstrakter Datentyp, der die Struktur für andere Typen vorgibt. \\
            \hline
            \textbf{ UnionTypeDefinition } & Ein Typ, der die Vereinigung verschiedener Typen ist. \\
            \hline
            \textbf{ EnumTypeDefinition } & Ein Typ, der nur eine feste Anzahl an vorher festgelegten Werten hat. \\
            \hline
            \textbf{ InputObjectTypeDefinition } & Zusammensetzung von ScalarTypes, um komplexere Eingabeargumente zu bilden.  \\
            \hline
        \end{tabularx}
        \caption{GraphQL Typen~\cite[vgl. 3.4 Types]{graphqlspecification}}
    \end{table}
\end{center}

Ein Typ hat einen einzigartigen Namen und definiert alle Informationen über sich, hierbei wird für jede Information ein Feld angelegt.
Das Feld setzt jeweils einen Typen um (InputObjectTypeDefinition ist dabei ausgeschlossen).
Ein sehr einfaches Schema wäre die Beziehung zwischen Büchern und Autoren.
Ein Buch hat einen Titel und einen Autor.
Der Autor hat einen Namen und ein Geburtsdatum.
Ein zugehöriges Schema für dieses Beispiel ist in Abbildung~\ref{buchauthorschema} dargestellt.

\begin{figure}[htb]
    \begin{center}
        \begin{lstlisting}[language=GraphQL]
        type Buch {
            title: String
            author: Autor
        }
        type Autor{
            name: String
            geburtsdatum: Date
        }
        \end{lstlisting}
    \end{center}
    \caption{Minimales Schema mit zwei Types}
    \label{buchauthorschema}
\end{figure}

Es lässt sich also festhalten, dass ein GraphQL-Typ immer als ein Tupel \verb+(Name, Felder)+ definiert wird, wobei
die Felder eine Liste an Tupeln \verb+(Feldname, Feldtyp, Datentyp)+ sind~\cite[vgl. 3.6 Objects]{graphqlspecification}.
Hierbei gelten Einschränkungen für die Elemente des Tupels.

\begin{center}
    \begin{description}
        \item[Feldname] ist ein eindeutiger Feldbezeichner
        \item[Feldtyp] gibt Einschränkungen vor, zum Beispiel nicht Null (durch \verb+!+), Listentyp (durch \verb+[]+) etc.
        \item[Datentyp] ist der explizite Typ den das Feld hat, kann Standarddatentyp oder anders definierter Type sein
    \end{description}
\end{center}

\subsection{Query}

Der Query-Type ist der Einstiegspunkt in alle Abfragen.
Da somit auch jeder Test im Query-Type beginnt, soll dieser hier näher betrachtet werden.
Abfragen können mit und ohne Eingabeparameter angegeben werden.
Informationen darüber finden sich im Schema.
Die definierten Anfragen haben, wie jeder Typ, einen eindeutigen Bezeichner, welcher dann in der zustellenden Abfrage benutzt wird.
Wenn das abgefragte Feld vom Typ $SCALAR$ ist, dann wird es direkt ausgegeben.
Ist das Feld vom Typ $OBJECT$, dann muss definiert werden, welche Felder des $OBJECTS$ gewünscht sind.
Eine korrekte Anfrage muss stets mindestens ein Feld des Typens $SCALAR$ enthalten.
Der Query-Type für das Schema aus Abbildung~\ref{buchauthorschema} kann umgesetzt werden durch Abbildung~\ref{minimalquerytype}.

\begin{figure}[htb]
    \begin{center}
        \begin{lstlisting}[language=GraphQL]
        type Query{
            # liste aller bücher
            getBooks: [Book]
            # ein zufälliges buch
            getBook: Book
            # author zum jeweiligen Buch
            getBookByTitle(String title): Author
        }
        \end{lstlisting}
    \end{center}
    \caption{Query Type für Buch und Autor}
    \label{minimalquerytype}
\end{figure}

Eine solche API ist in der Lage, drei verschiedene Anfragen zu beantworten.
Es muss beachtet werden, dass die Anfragen alle zwingend einen $SCALAR$ Type aufweisen müssen.
Somit ist die Anfrage \verb+getBookByTitle("Beispieltitel"){ author }+ nicht gültig und muss erweitert werden zu \verb+getBookByTitle("Beispieltitel"){ author{ name } }+.

\subsection{Resolver}
\label{resolver}

Bisher wurde die Strukturierung und Typisierung von GraphQL und den zugrundeliegenden Daten dargestellt.
Es ist noch unklar, wie GraphQL die zugrundeliegenden Daten abfragt und woher diese kommen.
Dies wird durch Resolver umgesetzt.
\\
\\
Ein Resolver ist in GraphQL eine Funktion, die zuständig für die Datenabfragen und Strukturierung ist~\cite[vgl.]{apolloqgl}.
Im Schema wird definiert, in welcher Form die Daten sein sollen.
Resolver sorgen dafür, dass die richtigen Daten in der richtigen Form zur Verfügung gestellt werden.
Die Resolver sind nicht, wie alle vorher benannten Teile von GraphQL offen einsehbar, sondern sind Funktionen einer Programmiersprache.
Sie setzen die Schnittstellenprogrammierung wie in Kapitel~\ref{api} angesprochen um.
Für jeden Typ im Schema muss ein Resolver implementiert werden.
Das umfasst insbesondere die Query, Mutation und Subscription-Typen, aber auch alle selbstdefinierten Typen~\cite[vgl.]{apolloqgl}.
Die konkrete Implementierung der Resolver hängt von verschiedenen Dingen ab, insbesondere jedoch davon, welcher GraphQL-Server genutzt wird.
Ein Resolver für die Query eines Buches anhand seines Titels mit ApolloServer in Javascript kann die Syntax aus Abbildung~\ref{simpleresolver} haben.

\begin{figure}[htb]
    \begin{verbatim}
    const resolvers = {
        Query: {
            book: (parent, args, context, info) => {
                return getBookByID(args.id);
            },
        },
    };
    \end{verbatim}
    \caption{Ein einfacher Resolver}
    \label{simpleresolver}
\end{figure}

Es ist zu beachten, dass alle Argumente, die mitgegeben werden, im \verb+args+ Argument gespeichert sind.
Die Funktion $getBookByID$ gibt ein dem Schema entsprechendes Json-Objekt zurück.
Da die Resolver konkrete Implementierungen außerhalb von GraphQL sind und die einzelnen Resolver untereinander aufrufen können, bedarf es hier
einer Reihe an Tests, damit die erwartete Funktionsweise des Programmcodes nachgewiesen werden kann.
Ein häufiger Fehler in GraphQL ist es, einzelne Attribute in einem Resolver zu vergessen.
Ein Entwickler sollte für jeden Resolver, der eine Funktion darstellt, einen beziehungsweise mehrere Tests zur Verfügung stellen.
Da sich die einzelnen Resolver aber auch untereinander aufrufen können, kann sich ein teils riesiger Testraum ergeben.
Hierzu mehr in Kapitel~\ref{abfrgraph}.
Um sicherzustellen, dass der Kombinationsaufruf der Resolver fehlerfrei ist, wird im Folgenden eine Methode entwickelt, welche gewährleistet,
dass die möglichen Kombinationen ausreichend durch Tests abgedeckt sind und so die Qualität und Zuverlässigkeit von GraphQL-APIs erhöht wird.
