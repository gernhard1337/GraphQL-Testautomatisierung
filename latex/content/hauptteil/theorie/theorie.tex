\chapter{Grundlagen / Theorie}

Das automatisierte Testen von GraphQL-API's erfordert ein spezifisches Domänenwissen in verschiedenen Teilbereichen der
Informatik und Mathematik. Dieses Domänenwissen wird in den folgenden Abschnitten aufgrundlage zweier Lehrbücher erarbeitet
und in Kontext gesetzt. Wissen über die Graphentheorie wird benötigt, da GraphQL eine Implementierung von graphenähnlichen Strukturen
ist und wir somit Algorithmen darauf anwenden können. Die mathematische Formalisierung hilft hierbei dann insbesondere
bei der Beweisführug für eine allgemeine Termination der zu entwickelnden Algorithmen.
Desweiteren ist es nötig sich bewusst zu machen, welche Arten des Testens von Software es gibt und warum wir bestimmte
Methoden hier eher nutzen als andere.
Im konkreten ist das Theorie-Kapitel so strukturiert, dass erst einmal die mathematischen Grundlagen der Graphentheorie
vermittelt werden und im Anschluss dazu wird eine Beziehung zwischen GraphQL & Graphentheorie hergestellt.
Mit der Beziehung können wir dann zeigen, dass Graphalgorithmen auch bei GraphQL anwendbar sind.
Mit diesen Grundlagen können wir dann zeigen, dass verschiedene Überdeckungsalgorithmen zur idealen Testgestaltung genutzt werden können
wobei hier natürlich definiert werden muss, was überhaupt eine Testüberdeckung ist und wann diese "ideal" ist.

