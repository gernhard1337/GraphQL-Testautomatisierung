\chapter{Grundlagen}
\label{theorie}

Das automatisierte Testen von GraphQL-APIs erfordert ein spezifisches Domänenwissen in verschiedenen Teilbereichen der
Informatik und Mathematik, insbesondere die Graphentheorie und das Softwaretesten.
Dieses Domänenwissen wird in den folgenden Abschnitten auf Grundlage zweier Lehrbücher~\cite{software-testing,graphentheorie} erarbeitet
und in Kontext gesetzt.
Die Graphentheorie wird benötigt, um die Struktur von GraphQL auf einer abstrakten Ebene besser verstehen zu können.
Außerdem setzen die in~\cite{software-testing} vorgestellten Überdeckungskriterien ein grundlegendes Wissen über Graphentheorie und Softwaretests voraus.