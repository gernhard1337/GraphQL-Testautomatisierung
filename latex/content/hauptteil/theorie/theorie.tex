\chapter{Grundlagen}
\label{theorie}

Das automatisierte Testen von GraphQL-APIs erfordert ein spezifisches Domänenwissen in verschiedenen Teilbereichen der
Informatik und Mathematik, insbesondere die Graphentheorie und das Softwaretesten.
Dieses Domänenwissen wird in den folgenden Abschnitten auf Grundlage zweier Lehrbücher~\cite{software-testing,graphentheorie} erarbeitet
und in Kontext gesetzt.
Wir benötigen die Graphentheorie um die Struktur von GraphQL auf einer abstrakten Ebene besser verstehen zu können.
Außerdem setzen die in~\cite{software-testing} vorgestellten Überdeckungskriterien ein grundlegendes Wissen über Graphentheorie und Softwaretests voraus.

% alter Text
%Wissen über die Graphentheorie wird benötigt, da GraphQL eine Implementierung von graphenähnlichen Strukturen
%ist und wir somit Algorithmen darauf anwenden wollen beziehungsweise können.
%Die mathematische Formalisierung hilft hierbei dann insbesondere
%bei der Beweisführug für eine allgemeine Termination der zu entwickelnden Algorithmen.
%Desweiteren ist es nötig sich bewusst zu machen, welche Arten des Testens von Software es gibt und wann man ''genug'' oder ''zu wenig'' getestet hat.
%Hierfür gibt es Metriken die auch insbesondere auf Graphen angewandet werden können und uns dann verraten können ob unsere Tests ausreichend sind.
%%Im konkreten ist das Theorie-Kapitel so strukturiert, dass erst einmal die mathematischen Grundlagen der Graphentheorie
%vermittelt werden und im Anschluss dazu wird eine Beziehung zwischen GraphQL \& Graphentheorie hergestellt.
%Mit der Beziehung können wir dann zeigen, dass Graphalgorithmen auch bei GraphQL anwendbar sind.
%Darauffolgend zeigen wir verschiedene Kriterien die eine Testüberdeckung ermöglichen würden und erklären die Unterschiede.