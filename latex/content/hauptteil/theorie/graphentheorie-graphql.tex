\section*{Zusammenhang Graphentheorie und GraphQL}

GraphQL erlaubt es uns, Typen zu definieren. Ein Type beinhaltet immer mindestens eine Property. Ein Type kann mit einem
Knoten eines Graphens gleichgesetzt werden. Man nehme zum Beispiel ein Buch und definiere hierfür einen Type

\begin{verbatim}
    type Book {
        id: Int
        title: String
    }
\end{verbatim}

Es exisitiert jetzt ein Objekt Book mit den Eigenschaften id als Integer und title als String.
Repräsentiert als Graphen hätten wir jetzt folgende Struktur

\begin{figure}[!htbp]
    \begin{center}
        \begin{tikzpicture}
            \node (1)[abstract, rectangle split, rectangle split parts=2] at (4,8)
                {
                    \textbf{Book}
                    \nodepart{second} \begin{description}
                                          \item[id:] Integer
                                          \item[title:] String
                                      \end{description}
                };
        \end{tikzpicture}
    \caption{Graph mit 1 Type}
    \end{center}
    \label{fig:1type}
\end{figure}

Eine Property kann entweder ein Standarddatentyp sein oder auf einen Type verweisen, dies kann der eigene Type oder auch
ein anderer Type sein. Fügen wir unserem Beispiel des Buches eine Property hinzu mit dem Type Author wobei der Author selbst
wie folgt definiert wird:

\begin{verbatim}
    type Book {
        id: Int
        title: String
        author: Author
    }
    type Author {
        id: Int
        name: String
    }
\end{verbatim}

So haben wir einen Graphen definiert, indem jedem Buch noch ein Author hinzugefügt wird. Hierdurch ergibt sich folgender Graph

\begin{figure2}
    \begin{center}
        \begin{tikzpicture}
            \node (1)[abstract, rectangle split, rectangle split parts=2] at (0,0)
                {
                \textbf{Book}
                \nodepart{second} \begin{description}
                                      \item[id:] Integer
                                      \item[title:] String
                                      \item[author:] Author
                \end{description}
                };
            \node (2)[abstract, rectangle split, rectangle split parts=2] at (10,0)
                {
                \textbf{Author}
                \nodepart{second} \begin{description}
                                      \item[id:] Integer
                                      \item[name:] String
                \end{description}
                };

            \draw [->] (1) edge node[top] { author } (2);
        \end{tikzpicture}
        \caption{Graph mit 2 Types}
    \end{center}
    \label{fig:2types}
\end{figure2}

Stand jetzt, sind noch keine zirkulären Abfragen möglich. Bauen wir nun ein zirkuläres Verhalten ein indem wir dem Type
Author noch die Property "written" (also alle seine geschriebenen Bücher) hinzufügen

\begin{verbatim}
    type Book {
        id: Int
        title: String
        author: Author
    }
    type Author {
        id: Int
        name: String
        written: [Books!]
    }
\end{verbatim}

so ergibt sich, dass wir einen zirkulären Graphen haben mit folgender Struktur

\begin{figure2}
    \begin{center}
        \begin{tikzpicture}
            \node (1)[abstract, rectangle split, rectangle split parts=2] at (0,0)
                {
                \textbf{Book}
                \nodepart{second} \begin{description}
                                      \item[id:] Integer
                                      \item[title:] String
                                      \item[author:] Author
                \end{description}
            };
            \node (2)[abstract, rectangle split, rectangle split parts=2] at (10,0)
                {
                \textbf{Author}
                \nodepart{second} \begin{description}
                                      \item[id:] Integer
                                      \item[name:] String
                                      \item[written:] [Book]
                \end{description}
            };

            \draw [ -> ] (1) edge[bend left=60] node[top] { author } (2);
            \draw [ -> ] (2) edge node[bottom] {written} (1);
        \end{tikzpicture}
        \caption{Graph mit 2 Types}
    \end{center}
    \label{fig:2typescirc}
\end{figure2}

Jetzt ist es möglich, zirkuläre Abfragen in folgender Form zu stellen:

\begin{verbatim}
    {
        book{
            author{
                book{
                    author{
                        name
                    }
                }
            }
        }
    }
\end{verbatim}

oder auch

\begin{verbatim}
    {
        book{
            author{
                name
                   }
            }
    }
\end{verbatim}

vorrausgesetzt der abzufragende Author hat nur ein Buch geschrieben.
Auf Grundlage dieser hier definierten Bildungsstrukturen können dann Graphen beliebiger Größe definiert werden.



