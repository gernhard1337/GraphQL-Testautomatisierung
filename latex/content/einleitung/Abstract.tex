\chapter{Kurzfassung}

Im Zuge der digitalen Transformation nimmt die Anzahl von Softwareanwendungen rasant zu~\cite{digitale-transformation}.
Insbesondere durch das Internet of Things und die generelle fortschreitende Vernetzung diverser Geräte nimmt Netzwerklast stark zu~\cite{iot-traffic}.
Bisheriger Standard für Kommunikation von Geräten über das Internet waren REST-APIs~\cite[vgl. Introduction]{hygraph}.
Diese haben gewisse Limitierungen wie zum Beispiel: Ineffizienz durch Overfetching/Underfetching, Anzahl an Requests, Versionierung, Komplexität und vieles mehr~\cite{hygraph}.
Mit der Veröffentlichung von GraphQL in 2015 wurde ein Konkurrent zu REST ins Leben gerufen, der diese Probleme beheben kann.
\\
\\
Durch GraphQL lässt sich insbesondere die Netzwerklast reduzieren, da eine GraphQL-Request, im Gegensatz zu REST, mehrere Anfragen in einer einzigen HTTP-Request zusammenfassen kann~\cite{awsrestgraphql}
und dabei auch nur die wirklich gewünschten Daten überträgt~\cite[vgl. Advantages of GraphQL APIs]{hygraph}.
Dadurch, dass jedoch die wachsende Anzahl von Softwareanwendungen auch in immer kritischere Bereiche des Lebens vordringt, ist es wichtig, die Qualität der Software sicherzustellen~\cite[S. 16]{software-testing}.
Eine Methodik zum Sicherstellen der korrekten Funktionalität von Software ist das Testen von Software im Sinne von Validierungstests, die sicherstellen sollen, dass die Software vorher definierte Szenarien nach Erwartung behandelt.
Für REST-APIs existieren zahlreiche Tools, die solche Validierungstests automatisch übernehmen können, wohingegen es noch einen Mangel an Tools dieser Art für GraphQL gibt~\cite[vgl. Introduction]{property-based-testing}.
\\
\\
Im Rahmen der internationalen Konferenz für Automatisierung von Softwaretests IEEE/ACM 2021 wurde ein Paper veröffentlicht, das eine Methode vorstellt, wie GraphQL-APIs mithilfe von
\textit{Property-based Tests}~\cite{property-based-testing} automatisch getestet werden können.
Property-based bezieht sich darauf, dass die Eigenschaften eines Objektes genutzt werden, um diese zu testen.
Diese Methode generiert, der Datenstruktur angepasste, zufällige Tests und bietet so eine Möglichkeit, Fehler zu entdecken, ohne ein tiefgreifendes Wissen
des zu testenden System zu besitzen~\cite[vgl. Proposed Method]{property-based-testing}.
Die zufallsbasierte Testgenerierung weist allerdings einige Schwachstellen auf.
So kann sie nicht garantieren, dass die generierten Tests zu jeder Zeit eine gute Abdeckung der GraphQL-API haben, denn es können einzelne Routen der API
komplett ausgelassen werden.
Es sind Testszenarios denkbar, die sehr viele False-Positives durchlassen und somit die Qualität der Software nicht ausreichend sicherstellen können.
GraphQL ermöglicht außerdem einen potenziell unendlichen Suchraum für die Tests.
Um den potenziell unendlichen Suchraum einzugrenzen wurde ein Rekursionslimit eingeführt, dass die Testlänge limitiert.
Diese Limitierung führt dazu, dass die Testabdeckung nur bis zu einem bestimmten Komplexitätsgrad des zu testenden System ausreichend ist.
Mit dieser Arbeit wurde ein anderer Ansatz für die automatisierte Testgenerierung untersucht, um die Testabdeckung verlässlich zu verbessern.
Ein Schwerpunkt der Arbeit lag hierbei darin, zuerst die theoretische Verknüpfung von GraphQL mit der Graphentheorie herzustellen.
\\
\\
Das gewonnene Wissen wird für eine Analyse verwendet, welches Graphabdeckungskriterium sich gut für Testgenerierung eignen würde.
Die erlangten Kenntnisse halfen bei der Entwicklung einer Methode zum Generieren von Tests.
Um die Methode zu validieren wurde ein Prototyp entwickelt, der die Methode automatisiert.
In zwei Experimenten konnte mit dem Prototyp gezeigt werden, dass die entwickelte Methode in der Lage ist, Fehler in GraphQL-APIs zu finden.
Insgesamt war es möglich, einen Punkt aus~\cite[VI. Future Work]{property-based-testing} zu erweitern, indem für eine bessere Graphabdeckung gesorgt wurde.
