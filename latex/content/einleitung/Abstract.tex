\chapter{Abstract}

Mit zunehmender Popularität von GraphQL ist es wichtig auch die Qualität von GraphQL-Api's zu validieren und verifizieren.
Aktuell gibt es aber noch Einschränkungen an Testtools für GraphQL-Apis, insbesondere im Bereich der automatischen Testanalyse.
Im Paper \" Automatic Property-based Testing of GraphQL APIs \" (Quelle hinzufügen) wurde sich mit einem automatischen
Testverfahren für GraphQL-API's beschäftigt.
Allerdings bietet diese Arbeit ein Verbesserungspotential, welches in dieser Arbeit untersucht und implementiert werden soll.
Im konkreten handelt es sich bei dem automatischen Testverfahren um ein Verfahren, das ein GraphQL-Schema aufgrund seiner
Struktur rekursiv untersucht und hieraus tests generiert.
Hierbei wird zwar Rücksicht auf die spezielle Graphstruktur genommen allerdings werden spezifische Grapheigenschaften nicht ausgenutzt, um die Tests zu verbessern.
Dabei sind diverse Arbeiten publiziert worden die sich mit Graphenüberdeckung beschäftigen welche auch
Algorithmen anwenden, die zyklische Strukturen gut überdecken.
Ziel dieser Arbeit ist es, dieses Wissen über Graphüberdeckung zu nutzen indem besagte GraphAlgorithmen
den rekursiven Suchalgorithmus des Papers ersetzen.
Hierdurch erhoffen wir uns eine Verbesserung der Test-Coverage.