\chapter{Abstract}

Mit zunehmender Popularität von GraphQL werden Tools benötigt, die Sicherheit, Zuverlässigkeit und korrekte Funktionalität
von GraphQL verifizieren und validieren können.
Durch die starke Typisierung von GraphQL lässt sich der Testraum jedoch sehr schön eingrenzen.
Aktuell gibt es aber noch Einschränkungen an Testtools für GraphQL-Apis, insbesondere im Bereich der automatischen Testanalyse.
Im Paper \" Automatic Property-based Testing of GraphQL APIs \" (Quelle hinzufügen) wurde sich mit einem automatischen
Testverfahren für Integrationstest für GraphQL-API's beschäftigt.
Hierbei wurde der Ansatz der zufallsbasierten Testgenerierung genutzt, um Tests zu generieren.
Die zufallsbasierte Testgenerierung weist allerdings einige Schwachstellen auf.
So kann Sie nicht garantieren, dass die API zu jeder Zeit eine gute Coverage hat.
Es sind Testszenarios denkbar, die sehr viele false-positives durchlassen \& somit die Qualität der Software nicht ausreichend sicherstellen können.
GraphQL ermöglicht außerdem einen potentiell unendlichen Suchraum für die Tests.
Um diesem Problem zu entgehen wurde ein einfaches rekursions-limit definiert was jedoch dazu führt, dass die Testabdeckung
wieder nicht garantiert sichergestellt werden kann.
Mithilfe von einem allgemeinen Graphalgorithmus, der sonst zur Kontrollflussgraphen Analyse benutzt wird, wollen wir zeigen,
dass es möglich ist mit einem iterativen Verfahren GraphQL zu testen.
Hierbei werden die Probleme des unendlichen Suchraumes und der zufälligen Testüberdeckung systematisch und effizient gelöst.
Um zu validieren, dass unsere Methode eine Verbesserung bietet werden einige Experimente durchgeführt in denen wir
verschiedene Metriken vergleich hierbei unter anderem: Anzahl der generierten Tests, Testabdeckung der Schnittstelle und
gefundene Fehler.





%Diese Arbeit hat allerdings einige Limitierungen wie zum Beispiel ein Rekursionslimit, dass die Größe der zu testenden
%API stark limitiert und einen Fokus auf zufallsbasierte Testgenerierung.
%In dieser Arbeit soll auf die Methode des \" Automatic Property-based Testing of GraphQL APIs \"
%eingegangen werden und mit einem anderen, strukturierteren Ansatz verbessert werden.
%Denn GraphQL definiert eine sehr stark typisierte Schnittstelle welche mithilfe von Graphalgorithmen untersucht werden kann.
%Ziel dieser Arbeit ist es, Graphüberdeckungen nach \cite{software-testing} zu untersuchen und Algorithmen zu implementieren
%die zu einer optimalen bzw hinreichenden Überdeckung eines Graphen führen.

%Hierzu soll der PrimePath Algorithmus, der klassischerweise für Graphüberdeckung in Kontrollflussgraphen verwendet wird, auf
%dieses Problem angepasst werden und evaluiert werden, inwiefern dieser zu einer Lösung des Problems beiträgt.
%Hierzu folgt am Ende ein Vergleich zwischen der ursprünglichen und neu entwickelten Methode.


%alter Text:
%Allerdings bietet diese Arbeit ein Verbesserungspotential, welches in dieser Arbeit untersucht und implementiert werden soll.
%Im konkreten handelt es sich bei dem automatischen Testverfahren um ein Verfahren, das ein GraphQL-Schema aufgrund seiner
%Struktur rekursiv untersucht und hieraus tests generiert.
%Hierbei wird zwar Rücksicht auf die spezielle Graphstruktur genommen allerdings werden spezifische Grapheigenschaften nicht ausgenutzt, um die Tests zu verbessern.
%Dabei sind diverse Arbeiten publiziert worden die sich mit Graphenüberdeckung beschäftigen welche auch
%Algorithmen anwenden, die zyklische Strukturen gut überdecken.
%Ziel dieser Arbeit ist es, dieses Wissen über Graphüberdeckung zu nutzen indem besagte GraphAlgorithmen
%den rekursiven Suchalgorithmus des Papers ersetzen.
%Hierdurch erhoffen wir uns eine Verbesserung der Test-Coverage.