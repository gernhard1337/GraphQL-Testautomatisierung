\chapter{Abstract}

Mit zunehmender Popularität von GraphQL ist es wichtig auch die Qualität von GraphQL-Api's zu testen.
Aktuell gibt es aber noch Einschränkungen an Testtools für GraphQL-Apis insbesondere im Bereich der
Testcoverage. Im Paper \"Automatic Property-based Testing of
GraphQL APIs" (Quelle hinzufügen) wurde sich mit einem automatischen Testverfahren für GraphQL-API's beschäftigt.
Allerdings bietet diese Arbeit ein Verbesserungspotential, welches in dieser Arbeit untersucht und implementiert werden soll.
Im konkreten handelt es sich bei dem automatischen Testverfahren um ein Verfahren, das ein GraphQL-Schema aufgrund seiner
Eigenschaften aufspaltet und Testet. Hierbei wird zwar Rücksicht auf die Graphstruktur genommen allerdings werden spezifische
Grapheigenschaften nicht ausgenutzt um die Tests zu verbessern. Ziel dieser Arbeit ist es, das Domänenwissen für Graphen zu nutzen
um eine automatische Testgenerierung zu verbessern und somit eine zunächst höhere Test-Coverage bieten zu können.
