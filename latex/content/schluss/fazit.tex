\chapter{Fazit}
\label{fazit}

Unsere Behauptung, dass wir die Methode des \textit{Property-based Testings} um einen besseren Pfadpfindungsalgorithmus
als ein simples, zufälliges, begrenztes raten verbessern können hat sich als wahr herausgestellt.
Indem wir einen theoretischen Rahmen geschaffen haben, der zeigt, dass Graphabdeckunskriteren auf ein GraphQL-Schema anwendbar sind,
konnten wir eine Methode entwickeln, die für die Testentwicklung hinreichend ist.
Die entwickelten Methoden liefern eine gute Grundlage um das automatisierte Testen von GraphQL weiter voranzutreiben.
Final entstand dann daraus ein Prototyp, der fähig ist, Fehler in GraphQL-APIs zu finden.
Dies wurde an zwei Beispielen gezeigt und nachgewiesen, indem wir Fehler finden konnten.
Es wurden Schwachstellen offen gelegt und Ansätze entwickelt, die zur Weiterarbeit anregen.
Verwandte Arbeiten wurden betrachtet und wir konnten dabei feststellen, dass ein Ansatz wie in dieser Arbeit vorher noch nicht betrachtet wurde.
Generell lässt sich sagen, dass GraphQL eine zurecht stetig wachsende Technologie ist und Arbeiten wie diese hier
dazu beitragen, dass die Popularität von GraphQL wächst da die bisherigen Nachteile von GraphQL gegenüber REST so stückweise aufgelöst werden
und die großen Vorteile herausstechen.
Für mich persönlich hat diese Arbeit ein tiefgreifendes Verständnis von GraphQL gebracht und ich werde diese Technologie in Zukunft wesentlich
häufiger Nutzen.

