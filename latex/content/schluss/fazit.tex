\chapter{Fazit}
\label{fazit}

Unsere Behauptung, dass wir die Methode des \textit{Property-based Testings} um einen besseren Pfadpfindungsalgorithmus
als ein simples, zufälliges, begrenztes raten verbessern können hat sich als wahr herausgestellt.
Indem wir einen theoretischen Rahmen geschaffen haben, der zeigt, dass Graphkriterien, insbesondere Graphcoveragekriterien,
auf ein GraphQL-Schema anwendbar sind, konnten wir eine Methode entwickeln, die für die Testentwicklung hinreichend ist.
Hierbei haben wir eigene Teile eingebracht aber auch Methoden des \textit{Property-based Testings} verwandt, wie zum Beispiel die Argumentgenerierung.
Final entstand dann daraus ein Prototyp, der fähig ist, Fehler in GraphQL-APIs zu finden.
Dies wurde an zwei Beispielen gezeigt und nachgewiesen, indem wir Fehler finden konnten.
