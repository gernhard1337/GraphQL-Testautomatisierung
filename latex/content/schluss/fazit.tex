\chapter{Fazit}
\label{fazit}

Die Behauptung, dass die Methode des \textit{Property-based Testings} um einen besseren Pfadfindungsalgorithmus als ein simples, zufälliges, begrenztes Raten verbessert werden kann, hat sich als wahr herausgestellt.
Indem ein theoretischer Rahmen geschaffen wurde, war es möglich, Graphabdeckungskriterien zu nutzen, um Tests für GraphQL zu automatisiert zu generieren.
Die entwickelten Methoden liefern eine gute Grundlage, um das automatisierte Testen von GraphQL weiter voranzutreiben.
Final entstand dann daraus ein Prototyp, der fähig ist, Fehler in GraphQL-APIs zu finden.
Dies wurde an zwei Beispielen gezeigt und nachgewiesen, indem Fehler gefunden werden konnten.
Es wurden Schwachstellen offengelegt und Ansätze entwickelt, die zur Weiterarbeit anregen.
Verwandte Arbeiten wurden betrachtet und dabei konnte festgestellt werden, dass ein Ansatz wie in dieser Arbeit vorher noch nicht betrachtet wurde.
Generell lässt sich sagen, dass GraphQL eine berechtigterweise stetig wachsende Technologie ist und Arbeiten wie diese hier
dazu beitragen, dass die Popularität von GraphQL wächst, da die bisherigen Nachteile von GraphQL gegenüber REST so stückweise aufgelöst werden
und die großen Vorteile herausstechen.

