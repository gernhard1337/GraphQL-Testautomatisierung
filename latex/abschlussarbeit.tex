\usepackage[T1]{fontenc}% Mustervorlage für Abschlussarbeiten am LS DBIS
% Author: Alexander Stahl (stahlale@b-tu.de)
% Zuletzt aktualisiert am 10.02.2022

% Hinweis: Die Daten zur Arbeit speichern Sie in der Datei "data.tex".

% Diese Datei enthält Einstellungen und Paketimporte.
% Zur besseren Ordnung sollten Sie weitere Einstellungen bzw. Importe dieser Datei anfügen.

% Paketimporte
\documentclass[11pt,a4paper,german]{scrreprt}
\usepackage[utf8]{inputenc}  % Encoding
\usepackage{babel}  % Unterstützung verschiedener Sprachen
\usepackage{graphicx}  % Einbindung von Grafiken
\usepackage{amsmath}  % Verschiedene Symbole
\usepackage{amssymb}  % Verschiedene Symbole
\usepackage{latexsym}  % Verschiedene Symbole
\usepackage{booktabs}  % Schönere Tabellen
\usepackage{multicol}  % Flexiblere Tabellen
\usepackage[printonlyused]{acronym}  % Abkürzungsverwaltung
\usepackage{hyperref}  % Klickbare URLs, Titel und PDF-Metadaten
\usepackage{lipsum}  % Platzhaltertext (kann entfernt werden)
\usepackage[backend=bibtex,style=ieee]{biblatex}  % Moderne Literaturverwaltung
\usepackage{tikz} % Graphen zeichnen
\usetikzlibrary{positioning,shapes,shadows,arrows}

% Einstellungen
\hypersetup{
    colorlinks,
    citecolor=black,
    filecolor=black,
    linkcolor=black,
    urlcolor=black
}

\makeatletter

\newcommand\frontmatter{
	\cleardoublepage
	\pagenumbering{roman}}

\newcommand\mainmatter{
	\cleardoublepage
	\pagenumbering{arabic}}

\newcommand\backmatter{
  \if@openright
    \cleardoublepage
  \else
    \clearpage
  \fi
   }

\makeatother

% Laden der Literaturdaten
\addbibresource{quellen.bib}

% Definition der Dokumentendaten
% Zur besseren Kapselung wurde die Eingabe Ihrer Daten in diese Datei ausgelagert.
% Bitte einfach ausfüllen.

\newcommand{\TitelDeutsch}{Integrationstesten von GraphQL mittels Prime-Path Abdeckung} % Hier zentral angeben
\newcommand{\TitelEnglisch}{Integration testing of GraphQL using Prime-Path Coverage} % Übersetzten Titel auch angeben
\newcommand{\ArbeitsTyp}{Masterarbeit} % unzutreffendes hier löschen
\newcommand{\AuthorName}{Tom Lorenz} % Hier ändern
\newcommand{\MatrikelNr}{3711679}
\newcommand{\Studiengang}{Informatik M. Sc.}
\newcommand{\DatumThemenausgabe}{16.05.2023}
\newcommand{\DatumAbgabe}{31.8.2023}
\newcommand{\BetreuerEins}{Prof. Dr. rer. nat. Leen Lambers}
\newcommand{\BetreuerZwei}{Prof. Dr. rer. nat. Gerd Wagner} % Bitte immer akademische Titel angeben
\newcommand{\GutachterEins}{M. Sc. Lucas Sakizloglou} % Bitte immer akademische Titel angeben
  % In dieser Datei die Daten der Arbeit eingeben.

\titlehead{{Brandenburgische Technische Universität Cottbus-Senftenberg\\ Institut für Informatik\\ Fachgebiet Praktische Informatik/Softwaresystemtechnik}}
\subject{Masterarbeit\\ \vspace{0.5cm}\includegraphics[width=0.5\textwidth]{img/btu-logo.pdf}}
\title{\TitelDeutsch \\ {\normalsize \TitelEnglisch}}
\author{\AuthorName \\ MatrikelNr.: \MatrikelNr \\Studiengang: \Studiengang}
\date{\textit{Datum der Themenausgabe: \DatumThemenausgabe} \\ \textit{Datum der Abgabe: \DatumAbgabe}}
\publishers{1. Betreuer: \BetreuerEins \\ 2. Betreuer: \BetreuerZwei}

\hypersetup{pdfauthor={\AuthorName}}
\hypersetup{pdftitle={\TitelDeutsch}}

\tikzstyle{abstract}=[rectangle, draw=black, rounded corners, fill=blue!40, drop shadow,
text centered, anchor=north, text=white, text width=3cm]
\tikzstyle{comment}=[rectangle, draw=black, rounded corners, fill=green, drop shadow,
text centered, anchor=north, text=white, text width=3cm]
\tikzstyle{myarrow}=[->, >=open triangle 90, thick]
\tikzstyle{line}=[-, thick] % Header beinhaltet alle Einstellungen und Pakete. Bei Bedarf erweitern.

\begin{document}
  \frontmatter % kennzeichnet den vorderen Teil der Arbeit
  \maketitle % Titelseite generieren

  % Eidesstattliche Erklärung über Selbstständigkeit
  \section*{Eidesstattliche Erklärung}
Der Verfasser erklärt, dass er die vorliegende Arbeit selbständig, ohne fremde Hilfe und ohne Benutzung anderer als der angegebenen Hilfsmittel angefertigt hat.
Die aus fremden Quellen (einschließlich elektronischer Quellen) direkt oder indirekt übernommenen Gedanken sind ausnahmslos als solche kenntlich gemacht.
Wörtlich und inhaltlich verwendete Quellen wurden entsprechend den anerkannten Regeln wissenschaftlichen Arbeitens zitiert.
Die Arbeit ist nicht in gleicher oder vergleichbarer Form (auch nicht auszugsweise) im Rahmen einer anderen Prüfung bei einer anderen Hochschule vorgelegt oder publiziert worden.
Der Verfasser erklärt sich zudem damit einverstanden, dass die Arbeit mit Hilfe eines Plagiatserkennungsdienstes auf enthaltene Plagiate überprüft wird.

\vspace{2cm}

\noindent
\begin{tabular}{lcl}
  .................................... & \hspace{.4\textwidth} & ....................................\\
  Ort, Datum &  & Unterschrift

\end{tabular}


  % Automatische Generierung verschiedener Verzeichnisse
  % Inhaltsverzeichnis
  \tableofcontents
  \mainmatter
  % Es folgt der Haupttext der Arbeit. Passen Sie die Überschriften der Kapitel entsprechend an.
  % Beachten Sie, dass die hier angegebene Gliederung nur ein Vorschlag ist.
  % Außerdem wird empfohlen, jedes Kapitel in eine eigene Datei auszulagern und hier nur mittels \input einzubinden.
  \chapter{Abstract}

Im Zuge der digitalen Transformation nimmt die Anzahl von Softwareanwendungen rasant zu und insbesondere durch das
Internet of Things und die generelle fortschreitende Vernetzung diverser Geräte nimmt in diesem Maße auch die Netzwerklast
zu.
Bisheriger Standard für Kommunikation von Geräten über das Internet waren REST-APIs diese haben jedoch gewisse
Limitierungen wie zum Beispiel: Ineffizienz durch Overfechting/Underfetching, Anzahl an Requests, Versionierung und Komplexität uvm.
Mit der Veröffentlichung von GraphQL in 2015 wurde ein Konkurrent zu REST in das Leben gerufen der diese Probleme beheben kann.
Durch GraphQL lässt sich insbesondere die Netzwerklast reduzieren da eine GraphQL-Request, im Gegensatz zu REST, mehrere Anfragen in einer
einzigen HTTP-Request zusammenfassen kann und dabei auch nur die wirklich gewünschten Daten überträgt.
Dadurch, dass jedoch die wachsende Anzahl von Softwareanwendungen auch in immer kritischere Bereiche des Lebens vordringt, ist es
enorm wichtig die Qualität der Software sicherzustellen.
Eine Methodik zum Sicherstellen der korrekten Funktionalität von Software ist das Testen von Software im Sinne von Validierungstests die sicherstellen sollen,
dass die Software vorher definierte Szenarien nach Erwartung behandelt.
Für REST-APIs existieren zahlreiche Tools die solche Validierungstests automatisch übernehmen können wohingegen es noch einen Mangel
an Tools dieser Art für GraphQL gibt.
Im Rahmen der IEEE/ACM 2021 wurde ein Paper veröffentlicht, dass eine Methode vorstellt wie GraphQL-APIs mithilfe von
"Property-based" Tests automatisch getestet werden können.
"Property-based" bzeieht sich darauf, dass die Eigenschaften eines Objektes genutzt werden um diese zu testen.
Diese Methode generiert zufällig Tests und bietet so eine Möglichkeit, Fehler zu entdecken und
generell erstmal eine Grundlage an Tests zu schaffen frei nach dem Motto: "lieber irgendwelche Tests als gar keine Tests".
Die zufallsbasierte Testgenerierung weist allerdings einige Schwachstellen auf.
So kann Sie nicht garantieren, dass die API zu jeder Zeit eine gute Coverage hat denn es können einzelne Routen der API
komplett ausgelassen werden.
Es sind Testszenarios denkbar, die sehr viele false-positives durchlassen \& somit die Qualität der Software nicht ausreichend sicherstellen können.
GraphQL ermöglicht außerdem einen potentiell unendlichen Suchraum für die Tests.
Um diesem Problem zu entgehen wurde ein einfaches rekursions-limit definiert was jedoch dazu führt, dass die Testabdeckung nur bis zu einem
bestimmten Grad überhaupt reichen kann.
Längere Pfade die mit GraphQL definiert werden können, werden deshalb niemals getestet.

Mit dieser Arbeit soll ein anderer Ansatz für die automatisierte Testgenerierung untersucht werden.
Hierbei soll untersucht werden, inwiefern Graphcoverage auf GraphQL angewendet werden kann zur Generierung von
Tests.
Mithilfe von einem allgemeinen Graphalgorithmus, der sonst zur Kontrollflussgraphen Analyse benutzt wird, wollen wir zeigen,
dass es möglich ist mit einem iterativen Verfahren GraphQL zu testen.
Wir versprechen uns von den Graphcoverage-Algorithmen einer verlässliche und sichere Generierung von Tests die
einerseits effizient jegliche Methoden abdecken und andererseits die Anzahl der Tests minimal hält.
Um die Methodik zu validieren werden am Ende beide Tools einem Experiment unterzogen um zu sehen ob die hier
vorgestellte Methode eine Verbesserung erzielen kann.


%Mit zunehmender Popularität von GraphQL werden Tools benötigt, die Sicherheit, Zuverlässigkeit und korrekte Funktionalität
%von GraphQL verifizieren und validieren können.
%Durch die starke Typisierung von GraphQL lässt sich der Testraum jedoch sehr schön eingrenzen.
%Aktuell gibt es aber noch Einschränkungen an Testtools für GraphQL-Apis, insbesondere im Bereich der automatischen Testanalyse.
%Im Paper \" Automatic Property-based Testing of GraphQL APIs \" (Quelle hinzufügen) wurde sich mit einem automatischen
%Testverfahren für Integrationstest für GraphQL-API's beschäftigt.
%Hierbei wurde der Ansatz der zufallsbasierten Testgenerierung genutzt, um Tests zu generieren.
%Die zufallsbasierte Testgenerierung weist allerdings einige Schwachstellen auf.
%So kann Sie nicht garantieren, dass die API zu jeder Zeit eine gute Coverage hat.
%Es sind Testszenarios denkbar, die sehr viele false-positives durchlassen \& somit die Qualität der Software nicht ausreichend sicherstellen können.
%GraphQL ermöglicht außerdem einen potentiell unendlichen Suchraum für die Tests.
%Um diesem Problem zu entgehen wurde ein einfaches rekursions-limit definiert was jedoch dazu führt, dass die Testabdeckung
%wieder nicht garantiert sichergestellt werden kann.
%Mithilfe von einem allgemeinen Graphalgorithmus, der sonst zur Kontrollflussgraphen Analyse benutzt wird, wollen wir zeigen,
%dass es möglich ist mit einem iterativen Verfahren GraphQL zu testen.
%Hierbei werden die Probleme des unendlichen Suchraumes und der zufälligen Testüberdeckung systematisch und effizient gelöst.
%Um zu validieren, dass unsere Methode eine Verbesserung bietet werden einige Experimente durchgeführt in denen wir
%verschiedene Metriken vergleich hierbei unter anderem: Anzahl der generierten Tests, Testabdeckung der Schnittstelle und
%gefundene Fehler.


%Diese Arbeit hat allerdings einige Limitierungen wie zum Beispiel ein Rekursionslimit, dass die Größe der zu testenden
%API stark limitiert und einen Fokus auf zufallsbasierte Testgenerierung.
%In dieser Arbeit soll auf die Methode des \" Automatic Property-based Testing of GraphQL APIs \"
%eingegangen werden und mit einem anderen, strukturierteren Ansatz verbessert werden.
%Denn GraphQL definiert eine sehr stark typisierte Schnittstelle welche mithilfe von Graphalgorithmen untersucht werden kann.
%Ziel dieser Arbeit ist es, Graphüberdeckungen nach \cite{software-testing} zu untersuchen und Algorithmen zu implementieren
%die zu einer optimalen bzw hinreichenden Überdeckung eines Graphen führen.

%Hierzu soll der PrimePath Algorithmus, der klassischerweise für Graphüberdeckung in Kontrollflussgraphen verwendet wird, auf
%dieses Problem angepasst werden und evaluiert werden, inwiefern dieser zu einer Lösung des Problems beiträgt.
%Hierzu folgt am Ende ein Vergleich zwischen der ursprünglichen und neu entwickelten Methode.


%alter Text:
%Allerdings bietet diese Arbeit ein Verbesserungspotential, welches in dieser Arbeit untersucht und implementiert werden soll.
%Im konkreten handelt es sich bei dem automatischen Testverfahren um ein Verfahren, das ein GraphQL-Schema aufgrund seiner
%Struktur rekursiv untersucht und hieraus tests generiert.
%Hierbei wird zwar Rücksicht auf die spezielle Graphstruktur genommen allerdings werden spezifische Grapheigenschaften nicht ausgenutzt, um die Tests zu verbessern.
%Dabei sind diverse Arbeiten publiziert worden die sich mit Graphenüberdeckung beschäftigen welche auch
%Algorithmen anwenden, die zyklische Strukturen gut überdecken.
%Ziel dieser Arbeit ist es, dieses Wissen über Graphüberdeckung zu nutzen indem besagte GraphAlgorithmen
%den rekursiven Suchalgorithmus des Papers ersetzen.
%Hierdurch erhoffen wir uns eine Verbesserung der Test-Coverage.
  %! Author = Tom
%! Date = 07.12.2022
\chapter{Einleitung}

In diesem Kapitel wird an das Thema und die Motivation dieser Arbeit herangeführt.
Außerdem wird definiert, welche Ziele diese Arbeit erreichen soll und eine grobe Übersicht über die
Kapitelstruktur gegeben.

\section{Motivation}

Mit einer steigenden Nutzung von GraphQL wird es immer wichtiger, geeignete Tests für GraphQL-API's zu entwickeln damit eine
gute Softwarequalität sichergestellt werden kann.
Tests können manuell geschrieben oder automatisch generiert werden.
Während bei Unit-Tests für einzelne Methoden der Programmierer frei entscheiden kann, ob er die Tests selbst schreiben will
oder doch eher von einem Tool generieren lassen will sind bei Integrations-Tests die Testräume teilweise so groß, dass ein
manuelles Schreiben dieser Tests einerseits fehleranfällig aber auch schlicht zu langwierig ist.
Für REST-APIs existieren schon automatische Integrationstesttools wie zum Beispiel: EvoMaster \cite{evo-master} , QuickREST \cite{karlsson2019quickrest} oder RESTTESTGEN \cite{rest-test-gen}.
GraphQL-APIs haben leider noch einen Mangel an solchen automatischen Testtools.
Im Rahmen der IEEE/ACM 2021 wurde mit ''Automatic Property-based Testing of GraphQL APIs''\cite{property-based-testing} eine Methode vorgestellt die diesen Mangel angehen soll.
Ergebnis der Arbeit war hierbei ein Prototyp der die vorgestellte Methode umsetzten sollte.
Die vorgestellte Methode bezieht sich auf ''Property-based Testing'' wobei diese gleichzusetzen ist mit ''Random Testing'' \cite{property-based-testing}[vgl. 2.B]
Durch die starke Typisierung und Schema-Definition, welche prinzipiell ein Graph ist, lässt sich in GraphQL
sehr einfach auswerten welche Anfragen nun zulässig sind.
Die hier vorgestellte Methode nutzt den Fakt, dass zum Beispiel alle möglichen Anfragen immer im Query-Knoten
beginnen müssen und somit alle weitergehenden Felder im Schema innerhalb des Query-Knoten definiert sein müssen.
Da die Definition des Schemas der eines Graphens entspricht ist jedoch der mögliche Suchraum potentiell unendlich da GraphQL
Zyklen innerhalb des Graphens erlaubt.
Der unendliche Suchraum wurde durch ein Rekursionslimit begrenzt allerdings wird so eine schlechtere Test-Coverage erreicht.
\newpage
Die bisher entwickelte Methode funktioniert auf folgende Weise:

\begin{center}
    \includegraphics[width=\textwidth,height=\textheight,keepaspectratio]{content/einleitung/toolchain}
    \caption{Methode von~\cite{property-based-testing}}
\end{center}

wobei neben dem vorher erwähnten Rekursionslimit außerdem Punkt 6. ''Verify response Properties'' kritisch
zu betrachten ist da die Auswertung wirklich nur auf die Properties schaut.
Dies bedeutet, dass ein zurückgegebens Objekt nur auf seinen Typ überprüft wird aber nicht, ob seine tatsächlichen Rückgabewerte, die exakt erwartet sind.
Hierdurch können false-positives entstehen.

Die vorgestellte Methode wollen wir verbessern durch Änderung einiger Ansätze.
Hierbei sollen Graphcoverage-Algorithmen zum Einsatz kommen, die mit iterativen Verfahren auch zyklische Graphen ideal
überdecken können und dabei immer verlässlich arbeiten, sodass die generierten Tests stets für vergleichbare Ergebnisse sorgen und nicht
davon abhängig sind, dass der Zufall eine gute Überdeckung liefert.
Zusammenfassend sei gesagt, dass bei der Testgenerierung und Testauswertung Verbesserungen möglich sind und dies Gegenstand dieser Arbeit sein soll.

\section{Umsetzung}

Zuallererst wird in dieser Arbeit etwas Theorie definiert und in Bezug gesetzt.
Wir beginnen damit die Graphentheorie als mathematisches Konzept zu definieren denn dieses liegt GraphQL zugrunde.
Darauffolgend kommt eine kleine Einführung in GraphQL und dann bilden wir schon die Schnittstelle von GraphQL zur Graphentheorie.
Sobald diese Verbindung erfolgt ist können wir uns dem eigentlichen Problem widmen: Wie kann man mithilfe von Graphcoverage-Algorithmen
Tests generieren?
Hierfür erfolgt ein letzter Theorie-Exkurs über Software-Tests.
Mit diesen Grundlagen schaffen wir es dann unsere Methode zu entwickeln und können Sie auch mit ''Automatic Property-based Testing of GraphQL APIs''\cite{property-based-testing}
vergleichen.
Unsere Methode wird konzeptionell vorgestellt und dann folgen einige Implementierungsdetails sowie ein Vergleich beider Tools durch Experimente.
Abschließen wird die Arbeit mit einem Ausblick für zukünftige Arbeit und einem Fazit über unsere erreichten Verbesserungen.
  % Beispiel: \input{kapitel1.tex}
  \section{Zusammenhang Graphentheorie und GraphQL}

Nun, da wir ein grundlegendes Wissen über Graphentheorie und GraphQL erlangt haben, müssen wir noch zeigen,
dass Graphentheorie auch anwendbar ist auf GraphQL. Ohne diese Verknüpfung könnten wir uns nicht sicher sein,
dass die zukünftig vorgestellten Algorithmen ausführbar sind für eine GraphQL-API.

\subsection{Typen als Knoten}

Wie in $4.2.1$ schon gezeigt, ermöglicht GraphQL das Erstellen von eigenen Typen.
Diese Typen repräsentieren einzelne Datenobjekte.
Somit liegt es nahe, dass die einzelnen Typen als Knoten eines Graphens repräsentiert werden können.
Bleiben wir bei vorigem Beispiel, so definieren wir einen Typen $Book$:

\begin{figure}[ht]
    \centering
    \begin{minipage}[b]{0.4\textwidth}
        \begin{verbatim}
        type Book {
            id: Int
            title: String
        }
        \end{verbatim}
    \end{minipage}
    \hfill
    \begin{minipage}[b]{0.4\textwidth}
        \centering
        \includegraphics[width=\textwidth,height=\textheight,keepaspectratio]{img/book}
        \caption{Book als Knoten}
        \label{fig:mein_bild}
    \end{minipage}
\end{figure}

\subsection{Felder als Kanten}

Der eben definierte Typ Book besitzt aktuell nur zwei Felder, id und title.
Beides sind Skalare-Datentypen und bilden somit keine ausgehenden Kanten, denn Kanten symbolisieren Beziehungen zwischen Objekten.
Wir können einen Graphen mit $G$ = $(V,E)$ wobei $E =\emptyset$ definieren.
Allerdings wäre dies sehr restriktiv da wir überhaupt keine Verbindungen haben würden und viele Vorteile von GraphQL würden wegfallen.
Es ist aber auch möglich, dass ein Type ein Feld definiert das nicht vom Skalaren Datentyp ist.
So können wir das Buch um einen Typen $Author$ erweitern wobei der Author gleich definiert wird wie in $4.2.1$.
Diese Typdefinitionen führen dann zu diesem Graphen:

\begin{center}
    \includegraphics[width=\textwidth,height=\textheight,keepaspectratio]{img/book-author}
\end{center}

Wir nutzen also die Typdefinitionen aus um die Beziehungen zwischen einzelnen Objekten zu bekommen.
Hierbei wird ein Typ dann mit einem anderen Typen verbunden, wenn dieser als Feld im Ursprungstyp vorkommt.
Man muss hierbei klar beachten, dass dies eine gerichtete Beziehung ist.
Eben erstellte Beziehung definiert nur, dass ein Book das Feld Author hat. Wir können nach aktuellem Stand nicht
die Bücher eines Autoren herausfinden da wir diese Kante nicht deklariert haben.
Dies ist allerdings möglich. Definieren wir den Author dann also wie folgt:

\begin{verbatim}
    type Book {
        id: Int
        title: String
        author: Author
    }
    type Author {
        id: Int
        name: String
        written: [Books!]
    }
\end{verbatim}

so ergibt sich folgender, zyklischer Graph:

\begin{center}
    \includegraphics[width=\textwidth,height=\textheight,keepaspectratio]{img/book-author-circle}
\end{center}

\subsection{der komplette Graph}

Wir wissen nun also wie ein GraphQL-Schema einen Graphen repräsentiert und können Algorithmen die auf Graphen funktionieren auch
auf einem GraphQL-Schema anwenden.
Abschließend zu diesem Kapitel sei gesagt, dass es in GraphQL allerdings nicht möglich ist, jeden einzelnen Knoten direkt
abzufragen.
Dies hängt stets von der Definition des Query/Mutation-Types ab, denn alle eingehenden Anfragen müssen vom Query-Type bzw. alle
eingehenden Veränderungen müssen über den Mutation-Type ablaufen.
Der Einfachheit halber werden wir uns jetzt erstmal dem Query-Type widmen.
Jede Anfrage die in GraphQL getätigt werden kann, hat ihren Ursprung im Query-Type.
Somit folgt auch, dass alle Routen stets im Query-Type entspringen müssen.
Dies bedeutet, dass nur Knoten mit Erreichbarkeit vom Query-Type überhaupt abgefragt werden können.
In diesem Beispiel:

\begin{figure}[ht]
    \centering
    \begin{minipage}[b]{0.4\textwidth}
        \begin{verbatim}
    type Query {
        book: Book
        author: Author
    }

    type Book {
        id: Int
        title: String
        author: Author
    }
        \end{verbatim}
    \end{minipage}
    \hfill
    \begin{minipage}[b]{0.4\textwidth}
        \begin{verbatim}
    type Author {
        id: Int
        name: String
        written: [Books!]
    }

    type Publisher {
        name: String
        published:  [Book]
    }
        \end{verbatim}
    \end{minipage}
\end{figure}


wäre es nicht möglich einen Knoten des Publishers abzufragen da dieser Knoten nicht erreichbar ist vom Query-Type.

  \chapter{Praxis}

\section{Umsetzung der Methode}




  \backmatter % kennzeichnet den hinteren Teil der Arbeit
  %! Author = Tom
\section{Glossar}

Im Text werden einige Fachbegriffe genutzt. Hier findet sich deren Erklärung

\begin{description}
    \item[Begriff] Erklärung
    \item[GraphQL] Waren-Managment-System; Ein System das das Lager verwaltet und die kompletten Betriebsprozesse eines Lagers abbilden kann
    \item[]
\end{description}


  \appendix % Anhang

  \cleardoublepage
  \addcontentsline{toc}{chapter}{Literaturverzeichnis}
  % \nocite{*}  % Hier das Kommentarzeichen entfernen, um alle Quellen auszugeben.
  \printbibliography[title={Literaturverzeichnis}]  % Generiert das Literaturverzeichnis

  \vfill
  \begin{center}
   \emph{Onlineressourcen wurden im Juli 2021 auf ihre Verfügbarkeit hin überprüft.}
  \end{center}

\end{document}
